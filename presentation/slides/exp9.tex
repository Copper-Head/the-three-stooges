\begin{frame}{\expix}
	\textbf{motivation:}
	\begin{itemize}
		\item activation sequences are continues signals
		\item why not using methods for signal analysis on it?
	\end{itemize}
	\textbf{procedure:}
	\begin{itemize}
		\item we decided to use an ERP signal analysis method
		\item general idea: compare averaged signals to noise to detect \textit{real events}
		\item feed the network $V$ artificial character sequences of length 2
		\item all sequences share our character of interest (COI) as first character, the second character is unique for each sequence
		\item the average of activations for these sequences is our COI's representation
		\item do the same for each other character of the vocabulary, but average over all gathered activations (noise representation)
		\item this experiment was performed for LSTM and simple recurrent on the \lk -dataset
	\end{itemize}
	\textbf{results:}
	\begin{itemize}
		\item strong differences between characters
		\item but how to interpret it?
	\end{itemize}
	\textbf{conclusion:}
	\begin{itemize}
		\item again, this method was performed for single neurons, we did not get an insight on groups of neurons which we need
	\end{itemize}
\end{frame}