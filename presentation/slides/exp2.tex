\begin{frame}{\expii}
	\textbf{motivation:}
	\begin{itemize}
		\item try to reproduce \citeauthor{karpathy2015visualizing}'s results on formal language (character level)
		\item do especially the cell activations show correlations (visual) with properties of the input text?
		\item apart from our questions this addresses even more, if it has learnt something and/or if it reveals this something in a way, such that we are able to observe it in the activations

		\vspace{1em}
		$\rightarrow$ if, where, how, (who)?
	\end{itemize}
\end{frame}
\begin{frame}{\expii}
	\textbf{procedure:}
	\begin{itemize}
		\item \citet{karpathy2015visualizing} trained their network on continuous sequences with truncated backpropagation through time 
		\item to imitate this behaviour with blocks without running in memory issues, we built a sequence generator, that learns fixed-length sequences (length 250) with batch size 1, while the initial state value for a sequence $i$ is the state's value after reading sequence $i-1$
	\end{itemize}
\end{frame}
\begin{frame}{\expii}
	\textbf{results:}
	\begin{itemize}
		\item activations of cells in 3rd layer showed reactions on indentation of C-code / spacial characters vs. ńon-white-space characters
		\item in addition the hinton diagrams suggest cells reacting on capital letters
		\item “parantheses-effect“ could not be reproduced
		\item generated C-code never showed perfect syntax, but was sometimes close
		\item perfect comment-syntax (but words were random letter combinations)
	\end{itemize}
\end{frame}
\begin{frame}{\expii}
	\textbf{conclusion:}
	\begin{itemize}
		\item method too complicated, in addition too long training times because batch-size was tight to 1		
		\item overgeneration of comments, natural language like variable names: comments in training data should have been removed, model had to learn two languages
		\item plots indicate correlations, but computation for single but consistently reoccuring events is difficult / impossible (to be shown later)
		\item correlation with indentation / spaces is basically an effect of character repitition
	\end{itemize}
\end{frame}